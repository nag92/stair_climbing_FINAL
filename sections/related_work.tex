\section{Related Work}


\subsection{Stair Climbing}

Stair climbing is a non-linear motion. Ascending stairs requires one foot to remain on the ground while the other foot swings through a trajectory\cite{hicks2012temporal}. Several studies have examined how people climb upstairs. Motion capture is one tool that can be used for recording human kinematics and kinetics. In  \cite{chalodhorn2007learning}, Chalodhorn \textit{et. al} used a model-free approach to teach a robot to walk. In  \cite{hu2014online}, Hu \textit{et. al} used quadratic programming to optimize the mapping of the markers to the robot. In both cases, the motion capture mapping was mapped from a human to a humanoid robot. This research was limited to the direct robot control, the robot was only able to do the exact motion of the human.

 In \cite{taskjointmocap}, Hu \textit{et. al} focused on online generations of trajectories. They formulated the control problem as a Quadratic Programming (QP) problem, and the authors used both Cartesian and joint data as the imitation criteria. The formulation as a QP problem allows for inequality and equality constraints for the knee velocity. This formulation resolves the conflicts between the joint space and the Cartesian imitation data. This method is limited to level ground walking. This process did not address the problem of abstracting the trajectories for different stair heights; it only addresses level ground walking and fully actuated systems.
 
 The work conducted in \cite{andriacchi1980study} and \cite{hicks2011lower} focused on the kinematics and dynamics of stair climbing by using a combination of motion capture, force plates, and IMU sensors. These studies have produced high-resolution joint angle data for level ground walking and stair climbing. This research gathered data and analysis, it was not applied robotics.

\subsection{Learning from Demonstrations}

Learning from Demonstration (LfD) is the process of transferring skills to a robot by demonstration. In \cite{siciliano2016springer}  \cite{kormushev2011imitation} \cite{calinon2007teacher} Calinon \textit{et al} outline the process of teaching a robot how to follow a trajectory. There are several steps in LfD: demonstration modality, motion primitives, and encoding methods. There are several steps in the LfD process, as stated below. 

\begin{enumerate}
    \item Recognition of the task 
    \item Encoding of the motion 
    \item Retrieval of the task 
    \item Reproduction of task 
\end{enumerate} 

There are two teaching modalities to teach a robot to follow a trajectory and to collect data. The first teaching modality is kinesthetic teaching. This modality involves a user moving a robot passively or running an impedance controller through a task while recording the joint positions and torques \cite{Calinon2018}. Researchers have applied kinesthetic teaching for upper limb robotic systems for industry and human-robot interfaces. This method could be used by developing a exoskeleton with built in sensors. However, this suite could alter the natural gait of the person by increasing the mass and inertia of their limbs. 


While this method is a natural method of teaching, it is not well suited for teaching lower limb movement because it would require directly moving a leg through a gait motion. It would be difficult to replicate the leg's motion by the manual moving of a person or robot. 

The second teaching modality is visual observation. In this teaching modality, visual sensors record the desired motions and intentions of a demonstration for mapping onto a robotic system \cite{CalinonLee19}. In visual observation, motion capture (mocap) is the standard method. The marker system allows for precise tracking of points on a person with millimeter accuracy \cite{ott2008motion}. This method has a wide adaptation and high accuracy for gait data collection\cite{ViconGaiting}.

There are several LfD methods for learning trajectories and building models based on the data from the two teaching modalities, including Locally Weighted Regression (LWR), Hidden Markov Models (HMM), and Gaussian Mixed Regression (GMR). They are all methods of encoding motion into basis functions for learning and reproduction. The methods are similar but contain their pros and cons. LWR is the simplest method, with HMM and GMR being extensions of this method. HMM and GMR place the radial basis functions (RBF) on the trajectory more effectively.  

Hidden Markov Models (HMM) combines temporal scaling and transition probabilities through a double stochastic process \cite{calinon2007learning}. HMM has built-in temporal scaling, meaning that it can deal with demonstrations that are not aligned. As stated by Calinon \textit{et. al}, the double stochastic process makes it difficult to retrieve smooth and continuous trajectories. It is desirable to have smooth and continuous trajectories for exoskeletons, therefore this method was rejected. 

The GMR method uses GMM to provide a more comprehensive approach to the RBF placement and allows for the encoding of multiple trajectories \cite{calinon2013compliant}. GMM is similar to K-means, however, it uses the Expectation Maximization (EM) algorithm to find the optimal placement of the RBF on the data set. GMR is used to regress over the RBFs to find an underlining model.  The demonstration for training the model must be temporally scaled using Dynamic Time Warping (DTW). GMR is computationally fast and produces smooth and continuous trajectories. The benefits of using GMR are the following \cite{Calinon}: 

% GMR provides a more comprehensive approach to the RBF placement and allows for the encoding of multiple trajectories, while DMP is limited to a single demonstration for training. In GMR, the demonstrations are encoded using Gaussian Mixture Models (GMM)\cite{calinon2013compliant} \cite{Statisticaldynamical}. The demonstration for training the model must be temporally scaled using Dynamic time warping (DTW). GMR is computationally fast and produces smooth and continuous trajectories. The benefits of using GMR are the following \cite{Calinon}: 

\begin{enumerate}  
    \item Allows encoding of local correlation between motion variables
    \item Provides a principled approach to estimate the parameters of the RBF 
    \item Reduces the number of RBF   
    \item Online estimation of the DMP parameters and model selection   
\end{enumerate}  

There are four steps for building the imitation model listed below. \autoref{fig:demostation} illustrates the workflow for training and reproduction.

\begin{enumerate}
    \item Collection of gait data
    \item Encoding of motion into RBF
    \item Learning of regression function
    \item Reproduction, and manipulation of the trajectories
\end{enumerate}

Dynamic Movement Primitives is a method of breaking down demonstrations into their fundamental building blocks. Movement primitives aim to encode trajectories into building blocks that can be rearranged and manipulated. DMP is a popular form of movement primitive \cite{ijspeert2013dynamical}. DMP works by creating a stable underlying model and generating a forcing function to drive the system. This is different than finding the average of the trajectory, because it allows for temporal and spatial manipulation. 

In the classical formulation of DMPs, the model places radial basis functions along the single demonstration to pull the system towards the goal. GMR/GMM replaces this method with a methodical placement of the RBFs and the use of multiple demonstrations to train the model. DMPs are considered one of the gold standards of learning and replicating human motion \cite{nakanishi2004learning}. This allows for the encoding of variations along the path allowing the model to adapt to different scenarios. 

Based on the above referenced research, there is a need to combine Stair Climbing research with Learning from Demonstration (LfD) methods. The best methods for this research are in \cite{andriacchi1980study} and \cite{hicks2011lower} along with GMR. This allows for robust data collection to be combined with a well studied framework of teaching a lower limb exoskeleton trajectories. These methods together allow a compilation of data to generalize a trajectory for stair climbing for people and stairs of different heights.