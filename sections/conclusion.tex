\section{Conclusion and Future Work} 
\label{sec:conclusion}
This paper showed how to build an imitation model by learning stair climbing trajectories using mocap and GMM/GMR. This work showed a trajectory generation of the joint motion for the exoskeleton's joint to be generalized so it can be used by people of all leg lengths on variable stair heights. Increasing the amount of subjects can improve the robustness of the model. To climb the stairs, the left and right feet take turns ascending to the same step. The trajectories were trained on subjects of different heights, ages, and genders. By training in task space, only the location of the toe is required. The joint angles are calculated using inverse kinematics and the height of the person wearing the exoskeleton. This can also be applied to a bipedal walking robot to train it to climb stairs. Future research can explore out-of-plane movement using a similar framework. 

The model can handle different stair heights by changing the start and goal inputs.  The model was optimized using the Bayesian information criterion to find the correct number of bins. This score ensured that the forcing function was not over or under fitted to the data set. The next step is to integrate a motion controller for the exoskeleton to generate the necessary torques. Integrating a sensor to detect the stair location will prevent collisions and automatically generate the start and goal positions for the trajectory generation.  

All data used in this paper is available here: \url{https://github.com/WPI-AIM/AIM_GaitData.git}
The code for extracting, analyzing, and modeling is located here: \url{https://github.com/WPI-AIM/AIM\_GaitAnalysisToolkit}



