

Lower limb assistive exoskeletons for rehabilitation is a growing field of study. They allow people with spinal cord injuries and strokes to stand upright and walk. Exoskeleton controls research typically focuses on trajectory generation, motion control over the trajectories, and balance and stability. This work focuses on the generalization of people's joint trajectories climbing a single stair, for an exoskeleton to be developed for people of all leg lengths and be used on variable stair heights. The motion of subjects of different heights, ages, and genders climbing a single stair was collected using motion capture. By using Gaussian mixed regression and Gaussian mixed models, these trajectories were combined to build a model that generalizes individuals with different leg lengths, genders, ages and stair heights. The variability of the subject's leg length is accounted for by tracking a marker on the toe of the subject and using inverse kinematics to calculate the joint angles. The analysis in this paper shows that joint angles can be generalized to climb various stair heights. In addition, an open-source database of motion capture climbing data and an open-source framework for generating joint trajectories has been made available for public use.


