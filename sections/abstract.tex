

Lower limb assistive exoskeletons for rehabilitation is a growing field of study. They allow people with spinal cord injuries and strokes to stand upright and walk. Exoskeleton controls research typically focuses on trajectory generation, motion control over the trajectories, and balance and stability. This work focuses on the generalization of people's joint trajectories climbing a single stair, for an exoskeleton to be developed for people of all leg lengths and be used on variable stair heights. The motion of subjects of different heights, ages, and genders climbing a single stair was collected using motion capture. By using Gaussian mixed regression and Gaussian mixed models, these trajectories were combined to build a model that generalizes individuals with different leg lengths, genders, ages and stair heights. The variability of the subject's leg length is accounted for by tracking a marker on the toe of the subject and using inverse kinematics to calculate the joint angles. The analysis in this paper shows that joint angles can be generalized to climb various stair heights. In addition, an open-source database of motion capture climbing data and an open-source framework for generating joint trajectories has been made available for public use.



% Rehabilitation exoskeletons are a growing field of study, and research shows they can provide benefits for people with spinal cord injuries and strokes. To be effective as a locomotive aid, lower-limb exoskeletons must accurately reproduce human gait over a range of terrain and obstacles commonly encountered during daily life. Exoskeletons' control is typically divided into two layers: trajectory generation and motion control over the trajectories. This work focuses on the former, the generation of the joint trajectories. In this paper, the motion of several subjects of different heights, ages, and genders climbing stairs was collected using motion capture. The variability of the subject's leg length is accounted for by tracking a marker on the toe of the subject then using inverse kinematics to calculate the joint angles. By using Gaussian mixed regression and Gaussian mixed models, these trajectories can be combined to learn a model that generalizes to individuals with different leg lengths and stair heights. This paper presents an open-source database of mocap climbing data, an open-source framework for generating joint trajectories for people with different leg lengths and stair heights, and verification of the framework using the provided data.




% Spinal cord injuries result in lower limb paralysis resulting in a reduced quality of life. Rehabilitation lower limb exoskeletons are a growing field of study, and research shows they provide relief for people with spinal cord injuries. These exoskeletons need to be able to replicate human motion to avoid putting extra stress on the joints. Exoskeletons control is divided into two fields of study: the dynamic control and trajectory generation. This work focuses on the latter, the generation of the joint trajectories.  The human motion of stair climbing was collected using a motion capture system. By exploiting learning by demonstration, this data can be used to learn and generalize. By using Gaussian mixed regression and Gaussian mixed models, these trajectories can be combined to learn a model. The model can then be used for people with different leg lengths. Additionally, this model can be used to climb upstairs from different heights. The generalization also allows for joint constraints and underactuated systems. This paper presents a model to generate joint trajectories that can be used for subjects of varying heights and can be dynamically changed on the fly to climb stairs of different heights. 