\section{Discussion} 
The trajectory of the foot for each subject follows a similar trajectory in the plane as referenced in \autoref{fig:stick}. The $Y$ position varied among the subjects and stair configurations. The subjects had different heights, which affects their stride length. This variation along the path is encoded during the GMM processes. The DMP formulation abstracts the start and goal of the trajectories. The locations of the start and goal positions have to be determined using the dynamics of the system. The $Z$ position varies very little between the subject. This is expected since each subject started on the ground and moved to the same height of the stair. The imitation model was able to replicate the trajectories from the demonstrations. 

The smoothing of the DTW trajectory is essential, as shown in \autoref{fig:DTW}. The warped trajectory contains jagged sections, which results in poor fitting and placement of the Gaussian. The demonstrations used to train the forcing function should be smooth to ensure a smooth forcing model. A polynomial was used to smooth the trajectory. 


An imitation model's accuracy can be increased by using more subject demonstrations from a single stair configuration to train a model. The cost of imitation was lower for the model trained with multiple demonstrations than a model trained with only a single demonstration. The imitation model can be lowered by increasing the number of demonstrations used to build the model. Using the BIC score, the optimal number of bins can be determined to train the model. It is important to use the optimal number of bins, or else the data will be skewed.  

This method can replicate and abstract trajectories so they can be used to climb stairs of different heights. The proximity and shape of the stairs can be found using a variety of sensors including: LIDARs, cameras, or proximity sensors. The starting and goal location must be predetermined and measured to ensure stability. The imitation model that was built was able to track trajectories from the start to the goal. Using inverse kinematics the joint angles are found to control the exoskeleton or bipedal system. The hip and knee trajectories track the measured joint angles using mocap. 


