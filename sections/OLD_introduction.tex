\section{Introduction} 

Rehabilitation exoskeletons are a growing field of study. Systems like the Rewalk \cite{esquenazi2012rewalk}, Esko \cite{mertz2012next} have presented promising results; their systems enable people with spinal cord injuries (SCI) and other neurological conditions to stand and walk with assistance. The trajectory generation is essential so that joints follow natural motions. The control along these generations is important to deal with the mass of the person and exoskeleton as well as the contact dynamics. Exoskeletons directly interface with a person's body. The control of the joint movement closely follows the user's natural joint movement if the joint motion is an unnatural excess strain on the person's joints. 

Therefore, the exoskeleton must move in an anatomically correct manner. Both gaiting and stair climbing are actives of daily living (ADL). Stair climbing is considered a hazardous form of locomotion \cite{HicksLittle2011LowerEJ}  due to the potential for the foot to collide with the stair if the trajectory is incorrect. To avoid these complications, a great deal of focus must be put on the generation of the foot trajectory.  

Two fields of research in lower-limb exoskeleton control are the design of the controller and the trajectory controller \cite{huang2016optimisation}. These two fields are highly related; the controller produces the torques, while the trajectory generation moves the joints and ensures that the placement of the foot is in a stable position.  

There is a significant work on teaching a robotic system on how to walk. Teaching a robot to follow a task is challenging; it involves balancing leg trajectory generation, single-leg support, and double leg support. The Zero Moment Point (ZMP) algorithm is at the core of most gait algorithms, a method that puts the center of mass within the support polygon of the robot \cite{kajita2003biped}. ZMP ensures that the robot does not produce a moment around its center, causing the robot to fall. In \cite{GaitSynthesis}, they mapped the Cartesian mocap markers locations from the human demonstration to the robot while ensuring that the foot remained within the support polygon.   

Additionally, in \cite{taskjointmocap}, they focused on online generations of trajectories. They formulated the control problem as a Quadratic Programming (QP) problem, and the authors used both Cartesian and joint data as the imitation criteria. The formulation as the QP problem allows for inequality and equality constraints for the knee velocity. This formulation resolves the conflicts between the joint space and the Cartesian imitation data. This method limited to level ground walking. This process did not address the problem of abstracting the trajectories for different stair heights; it only addresses level ground walking and fully actuated systems.    

Abled-Bodied people have the innate ability to adapt and adjust to their environment.  For example, when they encounter stairs of different heights, they can ascend them with ease. Exoskeletons need to have the same ability to adapt to different stair heights. This paper presents a model to allow exoskeletons to adapt its motion to different stair heights. To train a model, a rich data set is required. A motion capture system is used to build a library of joint kinematics during stair ascent.

In order to increase the data in the community and prevent other researchers from needing to conduct a time-consuming trial involving expensive equipment, all data used is available \href{https://github.com/WPI-AIM/AIM_GaitData.git}{here} This data is then used to train an imitation model by using Gaussian Mixed Models (GMM) for encoding and Gaussian Mixed Regression (GMR) extraction the motion primitives and Dynamic Motion Primitives (DMPs) for manipulating the start and goal of the trajectories. This model will not need to be retrained for each individual and allows for stair height adaption.